\documentclass[a4paper,parskip]{scrartcl}

\usepackage{lmodern}
\renewcommand*\familydefault{\sfdefault}
\usepackage[T1]{fontenc}

\usepackage[ngerman]{babel} %german english spell checking
\usepackage[utf8]{inputenc} %allows ä,ö,ü and others
\usepackage{multicol} %allows multiple columns
\usepackage{enumitem} %allows to change enumerate style
\usepackage[colorlinks=true,pdfborder={0 0 0},urlcolor=cyan]{hyperref} %allows links
\usepackage{verbatim} %allows code
\usepackage{graphicx} %allows images
\usepackage{amssymb} %allows symbols for number sets

\setlist{nosep} %smaller lists, less space between lines

\title{Arbeitsjournal Projekt1}
\author{Dominik Meister, Lorenz Rasch}

\begin{document}
\maketitle
\section{Besprechung 27.02.2019}
Anwesend an der Besprechung war:"Dominik Meister, Lorenz Rasch und Olivier Biberstein". 
\subsection{Besprechung}
Die erste Besprechung des Projekts war wie ein Kickoff. Herr Biberstein hat uns kurz einen
Überblick verschafft, wie er das Projekt Mindmap versteht. Ebenfalls haben wir uns auf eine Besprechungsperiode geeinigt. Ebenfalls besprochen wurde ein erster Entwurf eines Domainmodels.
\subsection{Ausblick}
Die nächste Besprechung findet am 06.03.2019 statt. Bis dann sollte das Projekt Gerüst fertiggestellt werden, dieses wird dann besprochen. Ebenfalls sollte das Domainmodel angepasst und eine Vision für das Projekt geschrieben werden.

\section{Besprechung 06.03.2019}
Anwesend an der Besprechung war:"Dominik Meister, Lorenz Rasch und Olivier Biberstein". 
\subsection{Besprechung}
In dieser Besprechung wurde das Projektgerüst welches von uns erstellt wurde besprochen, an der Struktur wurden direkt an der Besprechung einige kleine Anpassungen vorgenommen. Ebenfalls wurde die Vision besprochen, diese sei wesentlich zu kurz und soll wie eine Einleitung in das Projekt sein. Ebenfalls wurde das Domainmodel besprochen, dieses ist gut könnte sich aber noch ändern im Verlaufe des Projekts.
\subsection{Ausblick}
Die nächste Besprechung findet am 13.03.2019 statt. Bis dann sollten die Verbesserungen in der Projektstruktur gemacht werden. Ebenfalls sollten folgende Punkte erledigt werden:"UseCases erfassen, Userstories erfassen, Domainmodel Beschreibung erstellen, Vision anpassen, Sprint-Planung überlegen.

\section{Besprechung 13.03.2019}
Anwesend an der Besprechung war:"Dominik Meister, Lorenz Rasch und Olivier Biberstein". 
\subsection{Besprechung}
In dieser Besprechung wurde die Dokumentation des Analyse Kapitels besprochen. Die Userstories sind zu wenig detailiert und nicht formal richtig, diese müssen bearbeitet werden. Ebenfalls sind Fehler bei den Use Cases und Szenarien. Allgemein sollte das Projekt besser formatiert werden, d.h. mehr Absätze machen damit das Dokument lesbarer wird.
\subsection{Ausblick}
Die nächste Besprechung findet am 20.03.2019 statt. Der Sprint-1 hat begonnen, dies bedeutet es wird begonnen mit dem Programmieren der Applikation. Auf nächstes mal soll ein Prototyp erstellt werden wie, das ganze ungefähr funktionieren soll. Ebenfalls sollten die oben genannten Punkte überarbeitet werden. Herr Biberstein legte uns noch nahe schneller zu arbeiten. 

\section{Besprechung 20.03.2019}
Anwesend an der Besprechung war:"Dominik Meister, Lorenz Rasch und Olivier Biberstein". 
\subsection{Besprechung}
In dieser Besprechung wurde der Stand nach einer Woche Sprint01 besprochen. Der Prototyp welcher das GUI und ein paar dummy Funktionen enthielt, wurde 
besprochen. Das Resultat war, dass das Konzept so passe und umgesetzt werden könne. Ebenfalls haben wir noch einmal über die Use Cases und User Stories gesprochen, die Use Cases müssen noch einmal angepasst werden.
\subsection{Ausblick}
Die nächste Besprechung findet am 03.04.2019 statt. Bis dann ist der Sprint01 fällig und wird dann in der nächsten Sitzung besprochen. Deshalb soll bis
dann am Programm und an der Dokumentation gearbeitet werden. 

\section{Besprechung 10.04.2019}
Anwesend an der Besprechung war:"Dominik Meister, Lorenz Rasch und Olivier Biberstein". 
\subsection{Besprechung}
In dieser Besprechung wurde der Stand nach einer Woche Sprint02 besprochen. Realisiert wurden dabei die 
grundlegenden Funktionen wie, Knoten erstellen, Knoten bewegen, Knoten editieren. Die Funktionalität des Verbindens wurde noch nicht richtig implementiert. Ebenfalls wurden noch einmal die Use Cases besprochen diese sind nun gut. 
\subsection{Ausblick}
Die nächste Besprechung findet am 24.04.2019 statt. Bis dann ist der Sprint02 fällig und wird dann in der nächsten Sitzung besprochen. Im Sprint02 sind die Funktionen zum Verbinden von Nodes und das Speichern und
Laden der Map zu tun. Herr Biberstein hat uns geraten besser zusammen zu arbeiten und die Arbeitsteilung
zwischen den Teamkameraden besser geteilt werden sollte.

\section{Besprechung 24.04.2019}
Anwesend an der Besprechung war:"Dominik Meister, Lorenz Rasch und Olivier Biberstein". 
\subsection{Besprechung}
In dieser Besprechung wurde der Stand nach dem Ende von Sprint03 besprochen. Die Funktionalitäten Verbinden laden und speichern sind im Programm und funktionieren. Zusätzlich wurden die Design Diagramme, SSD und SD besprochen. Diese müssen noch einmal überarbeitet werden.
\subsection{Ausblick}
Die nächste Besprechung findet am 08.05.2019 statt. An diesem Punkt sind wir 2 Wochen in Sprint03. Bis dann müssen die SSD und SD Diagramme überarbeitet werden und die zusätzlichen Features für Sprint03 implementiert werden. Das Ziel wäre es dann mit den Features für Sprint03 fertig zu sein, damit noch
genügend Zeit für eventuelle Bugs und anderes vorhanden ist.


\end{document}