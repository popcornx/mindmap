\chapter{Design}
\label{chap:design}

\section{System Sequence Diagram}
\begin{figure}[H]
	\centering
		\includegraphics[scale=0.7]{images/SSD.png}
	\caption{System Sequence Diagramm}
	\label{fig:domain_model}
\end{figure}
\label{sec:system_sequence_diagram}
Wie beim Diagramm zu sehen ist, übergibt der Benutzer dem Programm primär Klicks auf bestimmte Objekte. Das Programm erstellt anhand der Koordinaten dann so die Nodes und Connections. Beim Speicher Prozess erhält der Benutzer eine XML Datei mit den Informationen zu seinem Mindmap. Beim Lade Prozess übergibt der Benutzer dieses dann, das Programm erstellt anhand der Informationen die Knoten auf der Map.


\section{Sequence Diagram}
\label{sec:sequence_diagram}
\begin{figure}[H]
	\centering
		\includegraphics[scale=0.6]{images/sequence.png}
	\caption{System Sequence Diagramm}
	\label{fig:domain_model}
\end{figure}

\section{UML Diagramme}
\label{sec:uml_diagramme}