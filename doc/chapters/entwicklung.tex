\chapter{Entwicklung}
\label{chap:entwicklung}

\section{Sprint 1}
\label{sec:sprint_1}
Sprint 1 dauert vom 13.03 - 03.04
\section{Sprint 2}
\label{sec:sprint_2}
Sprint 2 dauert vom 03.4 - 24.04
\section{Sprint 3}
\label{sec:sprint_3}
Sprint 3 dauert vom 24.04 - 15.05
\section{Sprint 4}
\label{sec:sprint_4}
Sprint 4 dauert vom 15.05 - 29.05 (Abgabe Code Teil)

\section{Programm ausführen}
\label{sec:run_program}

\subsection{Tools and Software}
\label{subsec:tools}
Unsere Mindmap Applikation benutzt folgende Software und Frameworks.
\begin{itemize}
\item Java jdk1.8.0\_{}191
\item JavaFX http://javafx.com/javafx/8.0.191
\item JavaFX Scene Builder 10.0.0
\end{itemize}

\subsection{Installation}
Um das Programm auszuführen muss zu erst das Repository per Git geklont werden.
\begin{verbatim}
git clone https://gitlab.ti.bfh.ch/rascl1/mindmap.git
\end{verbatim}
Wenn das Projekt in ein IDE importiert wird, muss der Ordner "{}MindMap"{} als Grundlage gewählt werden.\\
Oder das Programm kann mit der \texttt{MindMap.jar} ausgeführt werden.