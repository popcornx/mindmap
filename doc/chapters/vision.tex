\chapter{Vision}
\label{chap:vision}

Jeder kennt diese Situation, man sitzt in einem Meeting und möchte kurz ein Brainstorming machen. Oder man möchte für die nächste Arbeit zu einem Thema eine Übersicht zu den Informationen erstellen. Eine der einfachsten Methoden so eine Übersicht darzustellen ist das Mindmap. Das Mindmap auch Gedankenlandkarte genannt ist eine Technik welche von Tony Buzan geprägt und entwickelt wurde. Das Mindmap basiert auf dem Prinzip der Assoziation. Dies kommt nicht von ungefähr, unser Gehirn arbeitet ebenfalls mit Assoziationen, es versucht ständig neue Informationen mit gewissen Kategorien und anderen Informationen zu verknüpfen. Das Mindmap basiert auf derselben Technik, deshalb fällt es uns auch sehr einfach ein Mindmap zu erstellen. Dieses zu erstellen ist jedoch mit den meisten Programmen eher mühsam, deshalb greift man auf den Stift und Papier zurück. Um hier Abhilfe zu schaffen kommt unser Projekt ins Spiel.

Ziel unseres Projektes, ist das Entwickeln einer Software mit welcher man möglichst einfach und schnell ein Mindmap erstellen kann. Dabei werden wir besonders Wert auf die Benutzerfreundlichkeit legen. Es sollte möglich sein in kürzester Zeit ein Mindmap zu erstellen. Das Programm sollte aber auch reif sein für komplexere Mindmaps. Deshalb wird das Programm auch Funktionen wie verschiedene Verbindungstypen und Farben unterstützen, um auch komplexere Mindmaps übersichtlich zu gestalten. Wichtige Funktionen welche das Programm ebenfalls bieten muss sind: das Speichern, Laden und Drucken/Exportieren der Mindmaps.\\
Das Programm wird als Standalone Software in JavaFX erstellt. JavaFX ist ein Framework von Oracle welches auf die Erstellung von GUI's und Multimedialen Inhalten spezialisiert ist.

Wir hoffen wir können durch dieses Projekt vielen Menschen helfen Ihre Ideen mithilfe unseres Programmes festzuhalten.
